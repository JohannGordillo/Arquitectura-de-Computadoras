\documentclass[12pt,letterpaper,fleqn]{article}

\usepackage[utf8]{inputenc}
\usepackage[spanish,es-nodecimaldot]{babel}
\usepackage{amsmath}
\usepackage{amssymb}
\usepackage{multicol}
\usepackage{graphicx}

\usepackage[dvipsnames]{xcolor}
\usepackage[most]{tcolorbox}

%\usepackage[table]{xcolor}
\usepackage{colortbl}

\usepackage{tabu}

\usepackage{pgfplots}
\pgfplotsset{width=10cm,compat=1.9}

\usepackage{mathtools}
\usepackage{tikz}

\usetikzlibrary{trees,positioning}
\usetikzlibrary{shapes.geometric, arrows}

% Definiendo tikzstyles.
\tikzstyle{startstop} = [rectangle, rounded corners, minimum width=3cm, minimum height=1cm, text centered, draw=black, fill=red!20]
\tikzstyle{decision} = [diamond, minimum width=3cm, minimum height=1cm, text centered, draw=black, fill=blue!20]
\tikzstyle{process} = [rectangle, minimum width=3cm, minimum height=1cm, text centered, draw=black, fill=green!20]
\tikzstyle{to} = [circle, radius=1, draw=black, fill=orange!20]
\tikzstyle{io} = [trapezium, minimum width=2.5cm, minimum height=1cm, text centered, draw=black, fill=yellow!20, trapezium left angle=60, trapezium right angle=120]
\tikzstyle{arrow} = [thick, ->, >=stealth]

\usepackage[top=1in, bottom=1in, left=1in, right=1in]{geometry}


\begin{document}

\begin{titlepage}
    \centering

    {\scshape\LARGE Universidad Nacional Autónoma de México \par}

    \vspace{1cm}
    {\scshape\Large Facultad de Ciencias\par}
    \vspace{1.5cm}

    \begin{center}
        \includegraphics[scale=.1]{../assets/img/logo.png}
    \end{center}

    \vspace{.8 cm}

    {\LARGE Práctica 01: \par}
    {\huge\bfseries Organización y Arquitectura de Computadoras \par}

    \vspace{0.5cm}
    \large{\itshape{Johann Ramón Gordillo Guzmán}} \small{ - 418046090} \\ \vspace{0.3cm}

    \vfill
	
    Proyecto presentado como parte del curso de
    \textbf{Organización y Arquitectura de Computadoras}
    impartido por el profesor \textbf{José de Jesús Galaviz Casas}. \par
    \vspace{0.1cm}
    {\large 19 de Agosto del 2019\par}
    \footnotesize{\textbf{Link al código fuente:} https://github.com/JohannGordillo/}
\end{titlepage}

\section*{Lenovo Ideapad G485-20136}
\noindent \textbf{Propietario:} Johann Ramón Gordillo Guzmán.\\
\subsection*{Procesador}
\renewcommand{\arraystretch}{2}
\setlength\arrayrulewidth{1pt}
\begin{table}[h]
    \begin{tabular}{|>{\columncolor[gray]{0.9}}p{8cm}|p{8cm}|}
        \hline
        Fabricante&AMD\\
        \hline
        Modelo&E - 300\\
        \hline
        Frecuencia&1.3 GHz\\
        \hline
        Núcleos&2\\
        \hline
        Arquitectura&64 bits\\
        \hline
    \end{tabular}
\end{table}
\subsection*{Memoria}
\renewcommand{\arraystretch}{2}
\setlength\arrayrulewidth{1pt}
\begin{table}[h]
    \begin{tabular}{|>{\columncolor[gray]{0.9}}p{8cm}|p{8cm}|}
        \hline
        Random Access Memory (RAM) & 2 GB\\
        \hline
        Caché del procesador: L1d (Datos) & 32 k\\
        \hline
        Caché del procesador: L1i (Instrucciones) & 32 k\\
        \hline
        Caché del procesador: L2 (Unificado) & 512 k\\
        \hline
    \end{tabular}
\end{table}
\subsection*{Disco}
\renewcommand{\arraystretch}{2}
\setlength\arrayrulewidth{1pt}
\begin{table}[h]
    \begin{tabular}{|>{\columncolor[gray]{0.9}}p{8cm}|p{8cm}|}
        \hline
        Capacidad & 500 GB\\
        \hline
        Tipo & ATA Disk\\   
        \hline
        Velocidad de Lectura & 103.7 MB/s\\
        \hline  
        Velocidad de Escritura & 60.0 MB/s\\
        \hline                       
    \end{tabular}
\end{table}
\newpage
\subsection*{Sistema Operativo}
\setlength\arrayrulewidth{1pt}
\begin{table}[h]
    \begin{tabular}{|>{\columncolor[gray]{0.9}}p{8cm}|p{8cm}|}
        \hline
        Distribución & Ubuntu 18.04.2 LTS\\
        \hline
        Versión del kernel & Linux 4.15.0-51-generic (x86\_64)\\   
        \hline
        Escritorio & GNOME 3.28.2\\
        \hline                
    \end{tabular}
\end{table}
\subsection*{Multimedia}
\renewcommand{\arraystretch}{2}
\setlength\arrayrulewidth{1pt}
\begin{table}[h]
    \begin{tabular}{|>{\columncolor[gray]{0.9}}p{8cm}|p{8cm}|}
        \hline
        Gráficos & AMD RADEON HD 6310 384 MB\\
        \hline
        Audio & HDA-Intel-HD-Audio Generic\\   
        \hline           
    \end{tabular}
\end{table}
\newpage
\subsection*{Resultados}
\renewcommand{\arraystretch}{2}
\setlength\arrayrulewidth{1pt}
\begin{table}[h]
    \begin{tabular}{|>{\columncolor[gray]{0.9}}p{8cm}|p{8cm}|}
        \hline
        GZIP Compression & 318.98 segundos\\
        \hline
        DCRAW & 318.98 segundos\\
        \hline
        FLAC Audio Encoding & 116.14 segundos\\
        \hline
        GNUPG & 78.62 segundos\\
        \hline
        Timed MAFFT Alignment & 54.11 segundos\\
        \hline
        Timed MRBAYES Analysis & 4,194.09 segundos\\
        \hline        
        Timed MPlayer Compilation & 843.01 segundos\\
        \hline    
        Timed PHP Compilation & 1,348.11 segundos\\
        \hline    
        REDIS LPOP & 315,524 requests/s\\
        \hline 
        REDIS SADD & 245,129 requests/s\\
        \hline   
        REDIS LPUSH & 118,089 requests/s\\
        \hline     
        REDIS GET & 287,732 requests/s\\
        \hline 
        REDIS SET & 212,032 requests/s\\
        \hline                                            
    \end{tabular}
\end{table}
\end{document}