% Importando paquetes.
\documentclass[12pt,letterpaper,fleqn]{article}
\usepackage[utf8]{inputenc}
\usepackage[spanish,es-nodecimaldot]{babel}
\usepackage{amsmath}
\usepackage{amssymb}
\usepackage{multicol}
\usepackage{graphicx}
\usepackage[table, svgnames, dvipsnames]{xcolor}
\usepackage{makecell, cellspace, caption}
\usepackage[dvipsnames]{xcolor}
\usepackage[most]{tcolorbox}
\usepackage{colortbl}
\usepackage{tabu}
\usepackage{mathtools}
\usepackage[top=1in, bottom=1in, left=1in, right=1in]{geometry}
\setlength\cellspacetoplimit{3pt}
\setlength\cellspacebottomlimit{3pt}

\begin{document}

% Portada del documento.
\begin{titlepage}
    \centering

    {\scshape\LARGE Universidad Nacional Autónoma de México \par}

    \vspace{1cm}
    
    {\scshape\Large Facultad de Ciencias\par}
    
    \vspace{1.5cm}

	% Logo UNAM Facultad de Ciencias.
    \begin{center}
        \includegraphics[scale=.17]{../assets/img/logo.png}
    \end{center}

    \vspace{1 cm}

    {\LARGE Práctica 1 \par}
    
    \vspace{.3cm}

    {\huge\bfseries Organización y Arquitectura de Computadoras \par}

    \vspace{2.4cm}
    
    \large{\itshape{Johann Ramón Gordillo Guzmán}} \small{ - 418046090} \\
    
    \vspace{0.5cm}
    
    \large{\itshape{José Jhovan Gallardo Valdez}} \small{ - 310192815} \\ 
    
    \vspace{0.3cm}

    \vfill
	
    Práctica presentada como parte del curso de
    \textbf{Organización y Arquitectura de Computadoras}
    impartido por el profesor \textbf{José de Jesús Galaviz Casas}. \par
    \vspace{0.2cm}
    {\large 21 de Agosto del 2019\par}
    \vspace{0.3cm}
    \footnotesize{\textbf{Link al código fuente:} https://github.com/JohannGordillo/}
\end{titlepage}

% Datos de la computadora.
\section*{Lenovo Ideapad G485-20136}
\noindent \textbf{Propietario:} Johann Ramón Gordillo Guzmán.\\

\subsection*{Procesador}
\renewcommand{\arraystretch}{2}
\setlength\arrayrulewidth{1pt}
\begin{table}[h]
    \begin{tabular}{|>{\columncolor[gray]{0.9}}p{8cm}|p{8cm}|}
        \hline
        Fabricante&AMD\\
        \hline
        Modelo&E - 300\\
        \hline
        Frecuencia&1.3 GHz\\
        \hline
        Núcleos&2\\
        \hline
        Arquitectura&64 bits\\
        \hline
    \end{tabular}
\end{table}

\subsection*{Memoria}
\renewcommand{\arraystretch}{2}
\setlength\arrayrulewidth{1pt}
\begin{table}[h]
    \begin{tabular}{|>{\columncolor[gray]{0.9}}p{8cm}|p{8cm}|}
        \hline
        Random Access Memory (RAM) & 2 GB\\
        \hline
        Caché del procesador: L1d (Datos) & 32 KB\\
        \hline
        Caché del procesador: L1i (Instrucciones) & 32 KB\\
        \hline
        Caché del procesador: L2 (Unificado) & 512 KB\\
        \hline
    \end{tabular}
\end{table}

\subsection*{Disco}
\renewcommand{\arraystretch}{2}
\setlength\arrayrulewidth{1pt}
\begin{table}[h]
    \begin{tabular}{|>{\columncolor[gray]{0.9}}p{8cm}|p{8cm}|}
        \hline
        Capacidad & 500 GB\\
        \hline
        Tipo & ATA Disk\\   
        \hline
        Velocidad de Lectura & 103.7 MB/s\\
        \hline  
        Velocidad de Escritura & 60.0 MB/s\\
        \hline                       
    \end{tabular}
\end{table}
\newpage

\subsection*{Sistema Operativo}
\setlength\arrayrulewidth{1pt}
\begin{table}[h]
    \begin{tabular}{|>{\columncolor[gray]{0.9}}p{8cm}|p{8cm}|}
        \hline
        Distribución & Ubuntu 18.04.2 LTS\\
        \hline
        Versión del kernel & Linux 4.15.0-51-generic (x86\_64)\\   
        \hline
        Escritorio & GNOME 3.28.2\\
        \hline                
    \end{tabular}
\end{table}

\subsection*{Multimedia}
\renewcommand{\arraystretch}{2}
\setlength\arrayrulewidth{1pt}
\begin{table}[h]
    \begin{tabular}{|>{\columncolor[gray]{0.9}}p{8cm}|p{8cm}|}
        \hline
        Gráficos & AMD RADEON HD 6310 384 MB\\
        \hline
        Audio & HDA-Intel-HD-Audio Generic\\   
        \hline           
    \end{tabular}
\end{table}
\newpage

% Resultados de las pruebas.
\subsection*{Resultados}
\renewcommand{\arraystretch}{2}
\setlength\arrayrulewidth{1pt}
\begin{table}[h]
    \begin{tabular}{|>{\columncolor[gray]{0.9}}p{8cm}|p{8cm}|}
        \hline
        GZIP Compression & 318.98 segundos\\
        \hline
        DCRAW & 304.55 segundos\\
        \hline
        FLAC Audio Encoding & 116.14 segundos\\
        \hline
        GNUPG & 78.62 segundos\\
        \hline
        Timed MAFFT Alignment & 54.11 segundos\\
        \hline
        Timed MRBAYES Analysis & 4,194.09 segundos\\
        \hline        
        Timed MPlayer Compilation & 843.01 segundos\\
        \hline    
        Timed PHP Compilation & 1,348.11 segundos\\
        \hline    
        REDIS LPOP & 315,524 requests/s\\
        \hline 
        REDIS SADD & 245,129 requests/s\\
        \hline   
        REDIS LPUSH & 118,089 requests/s\\
        \hline     
        REDIS GET & 287,732 requests/s\\
        \hline 
        REDIS SET & 212,032 requests/s\\
        \hline                                            
    \end{tabular}
\end{table}

\newpage
%%%%%%%%%%%%%%%%%%%%%%%%%%%%%%%%%%%%%%%%%%%%%%%%%%%%%%%%%%%%%%%%%%%%%%%%%%%%%%%%%%%%
% Datos de la computadora.
\section*{Macbook Pro 2012 (Oracle Virtual Machine)}
\noindent \textbf{Propietario:} José Jhovan Gallardo Valdez.\\

\subsection*{Procesador}
\renewcommand{\arraystretch}{2}
\setlength\arrayrulewidth{1pt}
\begin{table}[h]
    \begin{tabular}{|>{\columncolor[gray]{0.9}}p{8cm}|p{8cm}|}
        \hline
        Fabricante&Intel\\
        \hline
        Modelo&i5-3210M\\
        \hline
        Frecuencia&2.50 GHz\\
        \hline
        Núcleos&2\\
        \hline
        Arquitectura&64 bits\\
        \hline
    \end{tabular}
\end{table}

\subsection*{Memoria}
\renewcommand{\arraystretch}{2}
\setlength\arrayrulewidth{1pt}
\begin{table}[h]
    \begin{tabular}{|>{\columncolor[gray]{0.9}}p{8cm}|p{8cm}|}
        \hline
        Random Access Memory (RAM) & 8 GB\\
        \hline
        Caché del procesador: L1d (Datos) & 256 KB\\
        \hline
        Caché del procesador: L1i (Instrucciones) & 256 KB\\
        \hline
        Caché del procesador: L2 (Unificado) & 3 MB\\
        \hline
    \end{tabular}
\end{table}

\subsection*{Disco}
\renewcommand{\arraystretch}{2}
\setlength\arrayrulewidth{1pt}
\begin{table}[h]
    \begin{tabular}{|>{\columncolor[gray]{0.9}}p{8cm}|p{8cm}|}
        \hline
        Capacidad & 500 GB\\
        \hline
        Tipo & Samsung SSD 850 EVO 500GB\\   
        \hline
        Velocidad de Lectura & 290 MB/s\\
        \hline  
        Velocidad de Escritura & 128 MB/s\\
        \hline                       
    \end{tabular}
\end{table}
\newpage

\subsection*{Sistema Operativo}
\setlength\arrayrulewidth{1pt}
\begin{table}[h]
    \begin{tabular}{|>{\columncolor[gray]{0.9}}p{8cm}|p{8cm}|}
        \hline
        Distribución & Ubuntu 18.04 LTS\\
        \hline
        Versión del kernel & Linux 4.14.0-generic (x86\_64)\\   
        \hline
        Escritorio &  GNOME 3.28 \\
        \hline                
    \end{tabular}
\end{table}

\subsection*{Multimedia}
\renewcommand{\arraystretch}{2}
\setlength\arrayrulewidth{1pt}
\begin{table}[h]
    \begin{tabular}{|>{\columncolor[gray]{0.9}}p{8cm}|p{8cm}|}
        \hline
        Gráficos & Intel HD Graphics 4000 1536 MB\\
        \hline
        Audio & Apple Audio Inc.\\   
        \hline           
    \end{tabular}
\end{table}
\newpage

% Resultados de las pruebas.
\subsection*{Resultados}
\renewcommand{\arraystretch}{2}
\setlength\arrayrulewidth{1pt}
\begin{table}[h]
    \begin{tabular}{|>{\columncolor[gray]{0.9}}p{8cm}|p{8cm}|}
        \hline
        GZIP Compression & 174 segundos\\
        \hline
        DCRAW & 230 segundos\\
        \hline
        FLAC Audio Encoding & 304.55 segundos\\
        \hline
        GNUPG & 51.45 segundos\\
        \hline
        Timed MAFFT Alignment & 63.16 segundos\\
        \hline
        Timed MRBAYES Analysis & 7159 segundos\\
        \hline        
        Timed MPlayer Compilation & 4220 segundos\\
        \hline    
        Timed PHP Compilation & 6094 segundos\\
        \hline    
        REDIS LPOP & 199779 requests/s\\
        \hline 
        REDIS SADD & 134151 requests/s\\
        \hline   
        REDIS LPUSH & 209228 requests/s\\
        \hline     
        REDIS GET & 597270 requests/s\\
        \hline 
        REDIS SET & 477972 requests/s\\
        \hline                                            
    \end{tabular}
\end{table}

% Sección de ejercicios de la práctica.
\newpage
\subsection*{1. Ejercicios}
\begin{enumerate}
\item Identifica cuáles de las pruebas miden el tiempo de respuesta y cuáles miden el rendimiento.\\\\
\noindent Las pruebas que miden el \textbf{tiempo de respuesta} son:\\
\noindent GZIP, DCRAW, FLAC Audio Encoding, GNUPG, Timed MAFFT Alignment, Timed MrBayer Analysis, Timed MPlayer Compilation y Timed PHP Compilation.\\\\
\noindent Las pruebas que miden el \textbf{rendimiento} son:\\
\noindent REDIS LPOP, REDIS SADD, REDIS LPUSH, REDIS GET y REDIS SET.\\

\item Usando la medida de tendencia central adecuada y tu reporte de resultados, calcula:
\begin{enumerate}

\item La medida de tiempo de respuesta.\\\\
\noindent Usando Media Armónica:\\\\
$\frac{8}{\frac{1}{318.98} + \frac{1}{304.55} + \frac{1}{116.14} + \frac{1}{78.62} + \frac{1}{54.11} + \frac{1}{4194.09} + \frac{1}{843.01} + \frac{1}{1348.11}}$ \\\\
$\approx 165.305$ \\\\

\item La medida de rendimiento.\\\\
\noindent Usando Media Aritmética:\\\\
$\frac{315524 + 245129 + 118089 + 287732 + 212032}{5} =$ \\\\
$\frac{11785.06}{5}$ \\\\
$\approx 235,701.2$ \\\\
\end{enumerate}

\item Una vez que tengas los reportes de tus compañeros, fija tu computadora como computadora de referencia. Calcula los tiempos normalizados y obtén la medida de tendencia central adecuada de cada una de las computadoras. Agrega los resultados al reporte.
\end{enumerate}
\newpage

\renewcommand{\arraystretch}{2}
\setlength\arrayrulewidth{1pt}
\begin{table}[h]
    \begin{tabular}{|>{\centering\arraybackslash}p{2.5cm}|>{\centering\arraybackslash}p{4.5cm}|>{\centering\arraybackslash}p{4.5cm}|>{\centering\arraybackslash}p{4.5cm}|}
        \hline
        \rowcolor{blue!20}
        Computadora & Fabricante & Modelo & Propietario\\
        \hline
        A & Lenovo & Ideapad G485-20136 & Johann\\   
        \hline       
        B & Oracle & VirtualBox v1.2 & Jhovan\\   
        \hline 
        C & Acer & Captain\_SK & Enrique\\   
        \hline          
        D & Oracle & VirtualBox v1.2 & Marcos\\   
        \hline                            
    \end{tabular}
    \caption*{Equipos de cómputo}
\end{table}

% Tabla con los resultados de las pruebas de rendimiento.
\subsubsection*{Pruebas de Rendimiento}
\renewcommand{\arraystretch}{2}
\setlength\arrayrulewidth{1pt}
\begin{table}[h]
    \begin{tabular}{|>{\centering\arraybackslash}p{2.5cm}|>{\centering\arraybackslash}p{3.26cm}|>{\centering\arraybackslash}p{3.26cm}|>{\centering\arraybackslash}p{3.26cm}|>{\centering\arraybackslash}p{3.26cm}|}
        \hline
        \rowcolor{green!20}
        Prueba & D & B & C & A\\
        \hline
        REDIS LPOP & 903499 & 199779 & 1412478.46 & 315524\\   
        \hline       
        REDIS SADD & 653868 & 134151 & 1181441.38 & 245129\\   
        \hline 
        REDIS LPUSH & 555148 & 209228 & 976264.92 & 118089\\   
        \hline          
        REDIS GET & 814157 & 597270 & 1432053.77 & 287732\\   
        \hline         
        REDIS SET & 529072 & 477972 & 1168740.87 & 212032\\   
        \hline                                    
    \end{tabular}
    \caption*{Resultados - Pruebas de Rendimiento}
\end{table}
\clearpage

% Tabla con los resultados de las pruebas de tiempo de respuesta.
\subsubsection*{Pruebas de Tiempo de Respuesta}
\renewcommand{\arraystretch}{2}
\setlength\arrayrulewidth{1pt}
\begin{table}[h]
    \begin{tabular}{|>{\centering\arraybackslash}p{2.5cm}|>{\centering\arraybackslash}p{3.26cm}|>{\centering\arraybackslash}p{3.26cm}|>{\centering\arraybackslash}p{3.26cm}|>{\centering\arraybackslash}p{3.26cm}|}
        \hline
        \rowcolor{green!20}
        Prueba & D & B & C & A\\
        \hline
        GZIP Compression & 60.92 & 174 & 58.51 & 318.98\\   
        \hline       
        DCRAW & 62.05 & 230 & 70.39 & 304.55\\   
        \hline 
        FLAC Audio Encoding & 17.39 & 304.55 & 18.27 & 116.14\\   
        \hline          
        GNUPG & 17.18 & 51.45 & 20.47 & 78.62\\   
        \hline         
        Timed MAFFT Alignment & 11.46 & 63.16 & 10.94 & 54.11\\   
        \hline 
		Timed MrBayes Analysis & 9.19 & 7159 & 1376.42 & 4194.09\\   
        \hline 
		Timed MPlayer Compilation & 173 & 4220 & 113.09 & 843.01 \\   
        \hline     
        Timed PHP Compilation & 348 & 6094 & 274.03 & 1348.11\\   
        \hline                                             
    \end{tabular}
    \caption*{Resultados - Pruebas de Tiempo de Respuesta}
\end{table}

\clearpage
\subsubsection*{Resultados finales}
\renewcommand{\arraystretch}{2}
\setlength\arrayrulewidth{1pt}
\begin{table}[h]
    \begin{tabular}{|>{\centering\arraybackslash}p{8cm}|>{\centering\arraybackslash}p{8cm}|}
        \hline
        \rowcolor{orange!20}
        Computadora & Tiempo de Respuesta \\
        \hline
        A & 165.305 $\frac{segundos}{benchmark}$ \\   
        \hline       
        B & 162.640 $\frac{segundos}{benchmark}$ \\   
        \hline 
        C & 33.402 $\frac{segundos}{benchmark}$ \\   
        \hline          
        D & 22.635 $\frac{segundos}{benchmark}$ \\  
        \hline                                           
    \end{tabular}
    \caption*{Tabla con los tiempos de respuesta por computadora}
\end{table}

\renewcommand{\arraystretch}{2}
\setlength\arrayrulewidth{1pt}
\begin{table}[h]
    \begin{tabular}{|>{\centering\arraybackslash}p{8cm}|>{\centering\arraybackslash}p{8cm}|}
        \hline
        \rowcolor{orange!20}
        Computadora & Rendimiento \\
        \hline
        A & 235,701.20 $\frac{requests}{segundo}$ \\   
        \hline       
        B & 323,680.00 $\frac{requests}{segundo}$ \\   
        \hline 
        C & 1,234,195.88 $\frac{requests}{segundo}$ \\   
        \hline          
        D & 691,148.80 $\frac{requests}{segundo}$ \\  
        \hline                                           
    \end{tabular}
    \caption*{Tabla con el rendimiento por computadora}
\end{table}

\clearpage
\renewcommand{\arraystretch}{2}
\setlength\arrayrulewidth{1pt}
\begin{table}[h]
    \begin{tabular}{|>{\centering\arraybackslash}p{2.5cm}|>{\centering\arraybackslash}p{3.26cm}|>{\centering\arraybackslash}p{3.26cm}|>{\centering\arraybackslash}p{3.26cm}|>{\centering\arraybackslash}p{3.26cm}|}
        \hline
        \rowcolor{red!20}
        Prueba & D & B & C & A\\
        \hline
        GZIP Compression & 0.191 & 0.545 & 0.183 & 1\\   
        \hline       
        DCRAW & 0.204 & 0.755 & 0.231 & 1\\   
        \hline 
        FLAC Audio Encoding & 0.150 & 2.622 & 0.157 & 1\\   
        \hline          
        GNUPG & 0.219 & 0.654 & 0.260 & 1\\   
        \hline         
        Timed MAFFT Alignment & 0.212 & 1.167 & 0.202 & 1\\   
        \hline 
		Timed MrBayes Analysis & 0.0022 & 1.707 & 0.328 & 1\\   
        \hline 
		Timed MPlayer Compilation & 0.205 & 5.01 & 0.134 & 1\\   
        \hline     
        Timed PHP Compilation & 0.258 & 4.52 & 0.203 & 1\\   
        \hline                           
        \rowcolor{yellow!20}
        Media Geométrica & 0.115 & 1.541 & 0.205 & 1\\
        \hline                  
    \end{tabular}
    \caption*{Resultados de las Pruebas de Tiempo de Respuesta normalizados tomando como referencia a la computadora A, con la media geométrica calculada posteriormente.}
\end{table}

\clearpage
\subsection*{2. Preguntas}
\begin{enumerate}
\item ¿Cuál computadora tiene el mejor tiempo de ejecución? Comparada con la computadora con la peor medida de tiempo de ejecución, ¿por qué factor es mejor la computadora?\\\\
\noindent La computadora con el mejor tiempo de ejecución es la computadora D, con:\\\\$22.635 \frac{segundos}{benchmark}$\\\\ y la computadora con la peor medida de tiempo de ejecución es la computadora A con:\\\\$165.305 \frac{segundos}{benchmark}$.\\\\
\noindent \textbf{El tiempo de ejecución de la computadora D es 7.3 veces mejor que el de la computadora A.}\\\\
\noindent El factor fue calculado mediante $\frac{A}{D}$ pues estamos hablando de medida de tiempo de ejecución y Lower Is Better.\\\\

\item ¿Cuál computadora tiene el mejor rendimiento? Comparada con la computadora con el peor desempeño, ¿por qué factor es mejor la computadora?\\\\
\noindent La computadora con el mejor rendimiento es la computadora C, con:\\\\$1,234,195.88 \frac{request}{segundo}$\\\\ y la computadora con el peor desempeño es la computadora A, con:\\\\$235,702.20 \frac{requests}{segundo}$.\\\\
\noindent \textbf{El rendimiento de la computadora C es 5.24 veces mejor que el de la computadora A.}\\\\
\noindent El factor fue calculado mediante $\frac{C}{A}$ pues estamos hablando de medida de rendimiento y Higher Is Better.
\newpage

\item De acuerdo con la computadora de referencia, ¿cuál computadora tiene el mejor desempeño y cuál computadora tiene el peor desempeño?\\\\
\noindent \textbf{La computadora con el mejor desempeño es la computadora D, y la computadora con el peor desempeño es la computadora B.}\\\\
Más aún, la computadora D tiene 13.4 veces mejor desempeño que la computadora B, pues $\frac{B}{D} = 13.4$.\\\\

\item De entre los atributos de cada máquina comparada, ¿cuáles resultan determinantes en la pérdida o ganancia de desempeño?\\\\
<<<<<<< HEAD
\noindent La computadora D fue mejor que las otras computadoras, ésta posee un procesador Intel i7 con 2 núcleos. Al ser un buen y moderno procesador, junto con los 8 GB de RAM que contiene la computadora D, se lleva por mucho a la computadora A que solo tiene 2 GB de RAM y un procesador AMD E-300. Aún así, los resultados arrojaron que la computadora con el peor desempeño fue la computadora B, que tiene 8 GB de RAM y un procesador Intel i5. Sin embargo, la computadora A tiene 2 núcleos y la computadora B únicamente 1.\\\\
Vemos que importa mucho el procesador para la ganancia de desempeño.
\\\\También cabe destacar que las computadoras B y D estaban corriendo ambas en una máquina virtual. Mientras que las computadoras A y C no.\\
=======
\noindent \textbf{El procesador.}
>>>>>>> master
\end{enumerate}
\subsection*{3. Punto extra}
\begin{enumerate}
\item Plantea un caso de uso para una computadora, es decir, describe una persona con alguna ocupación que requiera ciertos aspectos particulares de una computadora.\\
De acuerdo a los requerimientos del usuario pondera los resultados de las pruebas y obtén la medida de desempeño de cada una de las computadoras de tu equipo, de modo que puedas decir que computadora sería mejor para este usuario.\\\\

\noindent \textbf{Caso de ejemplo}\\\\
\noindent Juan es empleado de un Oxxo cercano a Ciudad Universitaria. Él se encarga de registrar los productos que compra cada cliente en el establecimiento, y de cobrarle el precio del producto.\\\\
Es por esto que Juan necesita una computadora que maneje bien las bases de datos, como por ejemplo REDIS. Juan realiza mucho la operación LPUSH, pues es la que ejecuta para añadir los productos a la lista de compras, seguida de SET y GET, las cuales ejecuta una vez por cliente, para crear y recuperar la lista de productos respectivamente.\\\\
Muy esporádicamente, utiliza LPOP para eliminar productos que agregó por accidente, pero esto no pasa frecuentemente.\\
Juan nunca utiliza SADD, puesto que las listas de compra no son conjuntos.
\end{enumerate}
\newpage

\noindent Si consideramos que cada cliente compra en promedio $5$ productos, y que Juan agrega productos accidentalmente a la lista una vez cada 20 compras, entonces podemos asignar los siguientes pesos:\\

% Tabla con los pesos para el punto extra.
\renewcommand{\arraystretch}{2}
\setlength\arrayrulewidth{1pt}
\begin{table}[h]
    \begin{tabular}{|>{\centering\arraybackslash}p{4cm}|>{\centering\arraybackslash}p{6cm}|>{\centering\arraybackslash}p{6cm}|}
        \hline
        \rowcolor{orange!20}
        Operacion & Por Compra & Peso \\
        \hline
        LPOP & 0.05 & 0.007092199 \\   
        \hline       
        SADD & 0 & 0 \\   
        \hline 
        LPUSH & 5 & 0.709219858 \\   
        \hline          
        GET & 1 & 0.141843972 \\   
        \hline         
        SET & 1 & 0.141843972 \\   
        \hline 
		& 7.05 & 1 \\   
        \hline                                           
    \end{tabular}
    \caption*{Tabla con los pesos}
\end{table}

<<<<<<< HEAD
\clearpage
\renewcommand{\arraystretch}{2}
\setlength\arrayrulewidth{1pt}
\begin{table}[h]
    \begin{tabular}{|>{\centering\arraybackslash}p{2.5cm}|>{\centering\arraybackslash}p{3.26cm}|>{\centering\arraybackslash}p{3.26cm}|>{\centering\arraybackslash}p{3.26cm}|>{\centering\arraybackslash}p{3.26cm}|}
        \hline
        \rowcolor{purple!20}
=======
\begin{table}[h!]
    \begin{tabular}{|>{\centering\arraybackslash}p{2.5cm}|>{\centering\arraybackslash}p{3.26cm}|>{\centering\arraybackslash}p{3.26cm}|>{\centering\arraybackslash}p{3.26cm}|>{\centering\arraybackslash}p{3.26cm}|}
        \hline
        \rowcolor{green!20}
>>>>>>> master
        Prueba & D & B & C & A\\
        \hline
        REDIS LPOP & 6407.79 & 1416.87 & 10017.58 & 2237.76\\   
        \hline       
        REDIS SADD & 0 & 0 & 0 & 0\\   
        \hline 
        REDIS LPUSH & 393721.98 & 148388.65 & 692386.47 & 83751.06\\   
        \hline          
        REDIS GET & 115483.26 & 84719.15 & 203128.19 & 40813.05\\   
        \hline         
        REDIS SET & 75045.67 & 67797.45 & 165778.85 & 30075.46\\   
        \hline                                    
<<<<<<< HEAD
    \end{tabular} 
    \caption*{Resultados \textbf{Ponderados} - Pruebas de Rendimiento }
\end{table}

\renewcommand{\arraystretch}{2}
\setlength\arrayrulewidth{1pt}
\begin{table}[h]
=======
    \end{tabular}
    \caption*{Resultados Ponderados- Pruebas de Rendimiento }
\end{table}

\begin{table}[h!]
>>>>>>> master
    \begin{tabular}{|>{\centering\arraybackslash}p{8cm}|>{\centering\arraybackslash}p{8cm}|}
        \hline
        \rowcolor{orange!20}
        Computadora & Rendimiento \\
        \hline
        A & 31375.47 $\frac{requests}{segundo}$ \\   
        \hline       
        B & 60464.42 $\frac{requests}{segundo}$ \\   
        \hline 
        C & 214262.22 $\frac{requests}{segundo}$ \\   
        \hline          
        D & 118131.74 $\frac{requests}{segundo}$ \\  
        \hline                                           
    \end{tabular}
<<<<<<< HEAD
    \caption*{Tabla con el rendimiento por computadora \textbf{(usando media ponderada)}}
\end{table}

\noindent La computadora que tiene mejor rendimiento es la C, lo cual podemos comprobar por medio de los cocientes $\frac{C}{A} = 6.82$, $\frac{C}{B} = 3.54$, $\frac{C}{D} = 1.81$. Ya que todos son mayores que 1.

\end{document}
=======
    \caption*{Tabla con el rendimiento por computadora (usando media ponderada)}
\end{table}
\begin{figure}[d]
La computadora que tiene mejor rendimiento es la C, lo cual podemos comprobar por medio de los cocientes $\frac{C}{A} = 6.82$, $\frac{C}{B} = 3.54$, $\frac{C}{D} = 1.81$.

Ya que todos son mayores que 1.
\end{figure}

\end{document}
>>>>>>> master
